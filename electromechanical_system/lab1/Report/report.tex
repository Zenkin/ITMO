\include{settings}

\begin{document}	% начало документа

% Титульная страница
\begin{titlepage}	% начало титульной страницы

	\begin{center}		% выравнивание по центру

		\large Санкт-Петербургский Национально Исследовательский Университет\\
		\large информационных технологий, механики и оптики \\
		\large Кафедра систем управления и информатики\\[6cm]
		% название института, затем отступ 6см
		
		\huge Технология изготовления элементов приборов и систем\\[0.5cm] % название работы, затем отступ 0,5см
		\large Отчет по лабораторной работе №5\\[0.1cm]
		\large Расчет режима резания при фрезеровании\\[1cm]
		\large Вариант №2\\[5cm]

	\end{center}


	\begin{flushright} % выравнивание по правому краю
		\begin{minipage}{0.25\textwidth} % врезка в половину ширины текста
			\begin{flushleft} % выровнять её содержимое по левому краю

				\large\textbf{Работу выполнили:}\\
				\large Зенкин А.М.\\
				\large Карпов К.В.\\
				\large {Группа:} P3335\\
				
				\large \textbf{Преподаватель:}\\
				\large Третьяков С.Д.

			\end{flushleft}
		\end{minipage}
	\end{flushright}
	
	\vfill % заполнить всё доступное ниже пространство

	\begin{center}
	\large Санкт-Петербург\\
	\large \the\year % вывести дату
	\end{center} % закончить выравнивание по центру

\thispagestyle{empty} % не нумеровать страницу
\end{titlepage} % конец титульной страницы

\vfill % заполнить всё доступное ниже пространство


% Содержание
\include{ToC}


\section{Цель работы}
Изучение математических моделей и исследование характеристик электромеханического объекта управления, построенного на основе электродвигателя постоянного тока независимого возбуждения.


\section{Варианты параметров}

$U_n = 110 [B], n_0 = 2500 [rot/min], I_n = 12 [A], M_n = 6.8 [H*m], R = 0.5 [Om], T_{ya} = 9 [ms], J_d=0.0015 [kg*m^2], T_y=5[ms], i_p = 40, J_m = 1.2[kg*m^2]$


\section{Ход выполнения работы}
\subsection{Изучить математические модели ЭМО и для полученного варианта задания рассчитать их параметры:}
\subsection{Составить схему моделирования ЭМО и получить графики переходных процессов напряжения, подаваемого на двигатель, тока якоря, скорости и углового положения вала нагрузки:}


\begin{figure}[H]
	\begin{center}
		\includegraphics[scale=0.37]{sim1}
		\caption{схема моделирования ЭМО} 
		\label{pic:pic_1} % название для ссылок внутри кода
	\end{center}
\end{figure}

\begin{figure}[H]
	\begin{center}
		\includegraphics[scale=0.25]{a}
		\caption{график моделирования a(t)} 
		\label{pic:pic_2} % название для ссылок внутри кода
	\end{center}
\end{figure}

\begin{figure}[H]
	\begin{center}
		\includegraphics[scale=0.25]{I}
		\caption{график моделирования I(t)} 
		\label{pic:pic_3} % название для ссылок внутри кода
	\end{center}
\end{figure}

\begin{figure}[H]
	\begin{center}
		\includegraphics[scale=0.25]{omega}
		\caption{график моделирования $\omega$(t)} 
		\label{pic:pic_4} % название для ссылок внутри кода
	\end{center}
\end{figure}

\begin{figure}[H]
	\begin{center}
		\includegraphics[scale=0.25]{U}
		\caption{график моделирования U(t)} 
		\label{pic:pic_5} % название для ссылок внутри кода
	\end{center}
\end{figure}

\subsection{Исследовать влияние момента сопротивления на вид переходных процессов:}

\begin{figure}[H]
	\begin{center}
		\includegraphics[scale=0.25]{alphaM}
		\caption{графики моделирования a(t)} 
		\label{pic:pic_2} % название для ссылок внутри кода
	\end{center}
\end{figure}

\begin{figure}[H]
	\begin{center}
		\includegraphics[scale=0.25]{IM}
		\caption{графики моделирования I(t)} 
		\label{pic:pic_3} % название для ссылок внутри кода
	\end{center}
\end{figure}
$tn_1$ = 0.17 s, I = 26 A;
$tn_2$ = 0.18 s, I = 19 A;
$tn_3$ = 0.19 s, I = 12.5 A;

\begin{figure}[H]
	\begin{center}
		\includegraphics[scale=0.25]{omegaM}
		\caption{графики моделирования $\omega$(t)} 
		\label{pic:pic_4} % название для ссылок внутри кода
	\end{center}
\end{figure}
$tn_1$ = 0.18 s, $\omega$ = 2.5 rad/s;
$tn_2$ = 0.17 s, $\omega$ = 2 rad/s;
$tn_3$ = 0.15 s, $\omega$ = 1.6 rad/s;


\begin{figure}[H]
	\begin{center}
		\includegraphics[scale=0.25]{UM}
		\caption{графики моделирования U(t)} 
		\label{pic:pic_5} % название для ссылок внутри кода
	\end{center}
\end{figure}

\subsection{Исследовать влияние момента инерции нагрузки на вид переходных процессов}

\begin{figure}[H]
	\begin{center}
		\includegraphics[scale=0.25]{i_2}
		\caption{графики моделирования I(t)} 
		\label{pic:pic_2} % название для ссылок внутри кода
	\end{center}
\end{figure}

$tn_1$ = 0.2 s, I = 0 A;
$tn_2$ = 0.13 s, I = 0 A;

\begin{figure}[H]
	\begin{center}
		\includegraphics[scale=0.25]{omega_2}
		\caption{графики моделирования $\omega$(t)} 
		\label{pic:pic_4} % название для ссылок внутри кода
	\end{center}
\end{figure}

$tn_2$ = 0.7 s, $\omega$ = 3.3 rad/s;
$tn_3$ = 0.15 s, $\omega$ = 3.3 rad/s;

\subsection{Собрать схему моделирования приближенной модели ЭМО и получить график переходного процесса скорости вращения нагрузки при моменте сопротивления = 0:}

\begin{figure}[H]
	\begin{center}
		\includegraphics[scale=0.25]{sim2}
		\caption{моделирование упрощенной схемы ЭМО} 
		\label{pic:pic_4} % название для ссылок внутри кода
	\end{center}
\end{figure}

\begin{figure}[H]
	\begin{center}
		\includegraphics[scale=0.25]{simplomega}
		\caption{график моделирования $\omega$(t) упрощенной модели} 
		\label{pic:pic_5} % название для ссылок внутри кода
	\end{center}
\end{figure}

\begin{figure}[H]
	\begin{center}
		\includegraphics[scale=0.25]{omega}
		\caption{график моделирования $\omega$(t)} 
		\label{pic:pic_5} % название для ссылок внутри кода
	\end{center}
\end{figure}

\newpage

\section{Вывод}
В данной лабораторной работе было проведено изучение математической модели и исследование характеристик электромеханического объекта управления, построенного на основе электродвигателя постоянного тока независимого возбуждения.
\end{document}
