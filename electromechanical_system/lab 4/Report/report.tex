\include{settings}

\begin{document}	% начало документа

% Титульная страница
\begin{titlepage}	% начало титульной страницы

	\begin{center}		% выравнивание по центру

		\large Санкт-Петербургский Национально Исследовательский Университет\\
		\large информационных технологий, механики и оптики \\
		\large Кафедра систем управления и информатики\\[6cm]
		% название института, затем отступ 6см
		
		\huge Технология изготовления элементов приборов и систем\\[0.5cm] % название работы, затем отступ 0,5см
		\large Отчет по лабораторной работе №5\\[0.1cm]
		\large Расчет режима резания при фрезеровании\\[1cm]
		\large Вариант №2\\[5cm]

	\end{center}


	\begin{flushright} % выравнивание по правому краю
		\begin{minipage}{0.25\textwidth} % врезка в половину ширины текста
			\begin{flushleft} % выровнять её содержимое по левому краю

				\large\textbf{Работу выполнили:}\\
				\large Зенкин А.М.\\
				\large Карпов К.В.\\
				\large {Группа:} P3335\\
				
				\large \textbf{Преподаватель:}\\
				\large Третьяков С.Д.

			\end{flushleft}
		\end{minipage}
	\end{flushright}
	
	\vfill % заполнить всё доступное ниже пространство

	\begin{center}
	\large Санкт-Петербург\\
	\large \the\year % вывести дату
	\end{center} % закончить выравнивание по центру

\thispagestyle{empty} % не нумеровать страницу
\end{titlepage} % конец титульной страницы

\vfill % заполнить всё доступное ниже пространство


% Содержание
\include{ToC}


\section{Цель работы}
Исследование принципов построения и свойств систем автоматического управления.

\section{Варианты параметров}

$U_n = 110 [B], n_0 = 2500 [rot/min], I_n = 12 [A], M_n = 6.8 [H*m], R = 0.5 [Om], T_{ya} = 9 [ms], J_d=0.0015 [kg*m^2], T_y=5[ms], i_p = 40, J_m = 1.2[kg*m^2]$


\section{Ход выполнения работы}

\subsection{Исследование системы с астатизмом нулевого порядка}

\subsubsection{Схема моделирования}
Схема моделирования представлена на рисунке \ref{pic:pic_1}.
\begin{figure}[H]
	\begin{center}
		\includegraphics[scale=0.35]{1}
		\caption{- схема моделирование} 
		\label{pic:pic_1} % название для ссылок внутри кода
	\end{center}
\end{figure}

\newpage

\subsubsection{Графики переходных процессов}
Графики переходных процессов для скорости нагрузки и ошибки для трех различных значений коэффициента регулятора приведены на рисунках \ref{pic:pic_2}-\ref{pic:pic_3}.
\begin{figure}[H]
	\begin{center}
		\includegraphics[scale=0.25]{K1_K9}
		\caption{- графки переходного процесса по скорости} 
		\label{pic:pic_2} % название для ссылок внутри кода
	\end{center}
\end{figure}

\begin{center}
$T_1 = 0.35\;c,\; T_2= 0.14\;c,\; T_3 =0.09\;c ;$
\end{center}

\begin{figure}[H]
	\begin{center}
		\includegraphics[scale=0.25]{K1_K9_E}
		\caption{- графики переходного процесса ошибки} 
		\label{pic:pic_3} % название для ссылок внутри кода
	\end{center}
\end{figure}

\begin{center}
$\varepsilon_{\text{уст}_1}=1.71,\;\varepsilon_{\text{уст}_2}=1.38,\;\varepsilon_{\text{уст}_3}=0.81$;
\end{center}


Определяем значение коэффициента регулятора, при котором САУ находится на границе устойчивости
\begin{center}
	$K = 14.6;$
\end{center}

\begin{figure}[H]
	\begin{center}
		\includegraphics[scale=0.25]{K_14_6}
		\caption{- графики переходного процесса по скорости} 
		\label{pic:pic_22} % название для ссылок внутри кода
	\end{center}
\end{figure}

\subsubsection{Подача на двигатель возмущение по моменту}
Подаем на двигатель возмущение по моменту M = 5 Нм при K = 2, график приведен на рисунке \ref{pic:pic_4}.
\begin{figure}[H]
	\begin{center}
		\includegraphics[scale=0.25]{K2_M_5}
		\caption{- графики переходного процесса скорости при возмущении по моменту} 
		\label{pic:pic_4} % название для ссылок внутри кода
	\end{center}
\end{figure}

\subsubsection{Подача на вход САУ линейно возрастающее воздействие}
Подаем на вход САУ линейно возрастающее воздействие и для одного из значений коэффициента регулятора получаем временную диаграмму для скорости нагрузки и ошибки. Графики переходных процессов приведены на рисунках \ref{pic:pic_5}-\ref{pic:pic_6}.\\
$y=2u,\; K = 5$
\begin{figure}[H]
	\begin{center}
		\includegraphics[scale=0.25]{K5_lin}
		\caption{- график переходного процесса по скорости} 
		\label{pic:pic_5} % название для ссылок внутри кода
	\end{center}
\end{figure}

\begin{figure}[H]
	\begin{center}
		\includegraphics[scale=0.25]{K5_lin_E}
		\caption{- графики переходного процесса ошибки} 
		\label{pic:pic_6} % название для ссылок внутри кода
	\end{center}
\end{figure}

\subsection{Исследование системы с астатизмом первого порядка}
Составляем схему моделирования САУ скоростью механической нагрузки, включающую пропорционально-интегральный регулятор и ОУ. Передаточная функция регулятора:
\begin{center}
	$W(s) = \dfrac{0.032285s+1}{0.032285s}$;
\end{center}

\subsubsection{Схема моделирования}
Схема моделирования представлена на рисунке \ref{pic:pic_7}.
\begin{figure}[H]
	\begin{center}
		\includegraphics[scale=0.3]{2}
		\caption{- схема моделирование} 
		\label{pic:pic_7} % название для ссылок внутри кода
	\end{center}
\end{figure}

\subsubsection{Графики переходных процессов}
Графики переходных процессов для скорости нагрузки и ошибки для трех различных значений коэффициента регулятора приведены на рисунках \ref{pic:pic_8}-\ref{pic:pic_9}.
\begin{figure}[H]
	\begin{center}
		\includegraphics[scale=0.22]{K1_K5_W}
		\caption{- графки переходного процесса по скорости} 
		\label{pic:pic_8} % название для ссылок внутри кода
	\end{center}
\end{figure}

\begin{center}
$T_1 = 0.2\;c,\; T_2= 0.25\;c,\; T_3 =0.33\;c ;$
\end{center}

\begin{figure}[H]
	\begin{center}
		\includegraphics[scale=0.25]{K1_K5_WE}
		\caption{- графики переходного процесса ошибки} 
		\label{pic:pic_9} % название для ссылок внутри кода
	\end{center}
\end{figure}

\begin{center}
$\varepsilon_{\text{уст}_1}=\varepsilon_{\text{уст}_2}=\varepsilon_{\text{уст}_3}=0.$;
\end{center}

\subsubsection{Подача на двигатель возмущение по моменту}
Подаем на двигатель возмущение по моменту M = 10 Нм при K = 3, график приведен на рисунке \ref{pic:pic_10}.
\begin{figure}[H]
	\begin{center}
		\includegraphics[scale=0.25]{K3_WM20}
		\caption{- графики переходного процесса скорости при возмущении по моменту} 
		\label{pic:pic_10} % название для ссылок внутри кода
	\end{center}
\end{figure}

\subsubsection{Подача на вход САУ линейно возрастающее воздействие}
Подаем на вход САУ линейно возрастающее воздействие и для одного из значений коэффициента регулятора получаем временную диаграмму для скорости нагрузки и ошибки. Графики переходных процессов приведены на рисунках \ref{pic:pic_11}-\ref{pic:pic_14}.\\
$y=2u,\; K = 5$\\

Для различных коэффициентов регулятора\\
\begin{figure}[H]
	\begin{center}
		\includegraphics[scale=0.25]{K3_K6_Wl}
		\caption{- график переходного процесса по скорости} 
		\label{pic:pic_11} % название для ссылок внутри кода
	\end{center}
\end{figure}

\begin{figure}[H]
	\begin{center}
		\includegraphics[scale=0.25]{K3_K6_WlE}
		\caption{- графики переходного процесса ошибки} 
		\label{pic:pic_12} % название для ссылок внутри кода
	\end{center}
\end{figure}

Для различной скорости изменения входного сигнала\\

\begin{figure}[H]
	\begin{center}
		\includegraphics[scale=0.25]{k2_k6_Wl}
		\caption{- график переходного процесса по скорости} 
		\label{pic:pic_13} % название для ссылок внутри кода
	\end{center}
\end{figure}

\begin{figure}[H]
	\begin{center}
		\includegraphics[scale=0.25]{k2_k6_WlE}
		\caption{- графики переходного процесса ошибки} 
		\label{pic:pic_14} % название для ссылок внутри кода
	\end{center}
\end{figure}

\subsection{Cхема моделирования САУ угловым положением механической нагрузки, включающую пропорциональный регулятор и ОУ}

\subsubsection{Схема моделирования}
Схема моделирования представлена на рисунке \ref{pic:pic_15}.
\begin{figure}[H]
	\begin{center}
		\includegraphics[scale=0.3]{3}
		\caption{- схема моделирование} 
		\label{pic:pic_15} % название для ссылок внутри кода
	\end{center}
\end{figure}

\subsubsection{Графики переходных процессов}
Графики переходных процессов для скорости нагрузки и ошибки для двух различных значений коэффициентов регулятора приведены на рисунках \ref{pic:pic_16}-\ref{pic:pic_17}.
\begin{figure}[H]
	\begin{center}
		\includegraphics[scale=0.25]{a5_10}
		\caption{- графки переходного процесса по скорости} 
		\label{pic:pic_16} % название для ссылок внутри кода
	\end{center}
\end{figure}

\begin{center}
$T_1 = 0.77\;c,\;T_2 =1.45\;c ;$
\end{center}

\begin{figure}[H]
	\begin{center}
		\includegraphics[scale=0.25]{a5_10_E}
		\caption{- графики переходного процесса ошибки} 
		\label{pic:pic_17} % название для ссылок внутри кода
	\end{center}
\end{figure}

\begin{center}
$\varepsilon_{\text{уст}_1}=\varepsilon_{\text{уст}_2}=0.$;
\end{center}

\subsubsection{Временные диаграммы при постоянной скорости и постоянном ускорении}
Получаем временную диаграмму при движении с постоянной скоростью и временную диаграмму при движении с постоянным ускорением. По графикам определяем время переходного процесса и установившееся значение ошибки. Графики смотреть на рисунках \ref{pic:pic_18} - \ref{pic:pic_21}

Движение с постоянной скоростью\\

\begin{figure}[H]
	\begin{center}
		\includegraphics[scale=0.21]{a}
		\caption{- график переходного процесса по углу} 
		\label{pic:pic_18} % название для ссылок внутри кода
	\end{center}
\end{figure}

\begin{figure}[H]
	\begin{center}
		\includegraphics[scale=0.21]{a_E}
		\caption{- график переходного процесса ошибки} 
		\label{pic:pic_19} % название для ссылок внутри кода
	\end{center}
\end{figure}

\begin{center}
$T = 0.43\;c;$
$\varepsilon_{\text{уст}}=0.155$;
\end{center}

Движение с постоянным ускорением\\

\begin{figure}[H]
	\begin{center}
		\includegraphics[scale=0.21]{aa}
		\caption{- график переходного процесса по углу} 
		\label{pic:pic_20} % название для ссылок внутри кода
	\end{center}
\end{figure}

\begin{figure}[H]
	\begin{center}
		\includegraphics[scale=0.21]{aa_E}
		\caption{- график переходного процесса ошибки} 
		\label{pic:pic_21} % название для ссылок внутри кода
	\end{center}
\end{figure}

\newpage

\section{Вывод}
В данной лабораторной работе было проведено исследование принципов построения и свойств систем автоматического управления. В чвстности САУ с астатизмом нулевого и первого порядка при различных входных сигналах и различный значениях коэффициента регулятора. Были получены коэффиценты регулятора, при которых система находится на границе устойчивости.
\end{document}