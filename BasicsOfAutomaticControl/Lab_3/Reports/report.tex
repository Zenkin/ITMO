\include{settings}

\begin{document}	% начало документа

% Титульная страница
\begin{titlepage}	% начало титульной страницы

	\begin{center}		% выравнивание по центру

		\large Санкт-Петербургский Национально Исследовательский Университет\\
		\large информационных технологий, механики и оптики \\
		\large Кафедра систем управления и информатики\\[6cm]
		% название института, затем отступ 6см
		
		\huge Технология изготовления элементов приборов и систем\\[0.5cm] % название работы, затем отступ 0,5см
		\large Отчет по лабораторной работе №5\\[0.1cm]
		\large Расчет режима резания при фрезеровании\\[1cm]
		\large Вариант №2\\[5cm]

	\end{center}


	\begin{flushright} % выравнивание по правому краю
		\begin{minipage}{0.25\textwidth} % врезка в половину ширины текста
			\begin{flushleft} % выровнять её содержимое по левому краю

				\large\textbf{Работу выполнили:}\\
				\large Зенкин А.М.\\
				\large Карпов К.В.\\
				\large {Группа:} P3335\\
				
				\large \textbf{Преподаватель:}\\
				\large Третьяков С.Д.

			\end{flushleft}
		\end{minipage}
	\end{flushright}
	
	\vfill % заполнить всё доступное ниже пространство

	\begin{center}
	\large Санкт-Петербург\\
	\large \the\year % вывести дату
	\end{center} % закончить выравнивание по центру

\thispagestyle{empty} % не нумеровать страницу
\end{titlepage} % конец титульной страницы

\vfill % заполнить всё доступное ниже пространство


% Содержание
\include{ToC}


\section{Цель работы}
Ознакомление с принципами построения моделей внешних воздействий — сигналов задания и возмущений.


\section{Варианты параметров}
$\phi = 24^\circ$, $f = 2 c^{-1}$\\
		
$\Delta=	4, V=2, F=10;$\\									
\section{Ход выполнения работы}

\subsection{Исследование командного генератора гармонического сигнала}

\subsubsection{Построение математической модели:}
\begin{equation}
	\begin{split}
&\omega=2\pi f=2\cdot 3.14\cdot 2=12.56;\\
&A=\frac{\tan(\phi)}{\omega}=\frac{tan(24^\circ)}{12.56}=0.0.354;\\
&z = \begin{bmatrix}
				z_1\\
				z_2
				\end{bmatrix}, G = \begin{bmatrix}
				0 & 1\\
				-157.7536 & 0
				\end{bmatrix}, H^T = \begin{bmatrix}
				1\\
				0
				\end{bmatrix};\\
	\end{split}				
\end{equation}
				
\subsubsection{Схема моделирования командного генератора:}
\begin{figure}[H]
	\begin{center}
		\includegraphics[scale=0.25]{s_1}
		\caption{- схема моделирования} 
		\label{pic:pic_1} % название для ссылок внутри кода
	\end{center}
\end{figure}

\subsubsection{Моделирование работы командного генератора:}
\begin{figure}[H]
	\begin{center}
		\includegraphics[scale=0.27]{g_1}
		\caption{- сигнал g(t) и синусоида} 
		\label{pic:pic_1} % название для ссылок внутри кода
	\end{center}
\end{figure}

\subsection{Исследование командного генератора сигнала с трапецеидальным графиком скорости}
\subsubsection{Построение математической модели:}
\begin{equation}
	\begin{split}
		&\Delta=	4, V=2, F=10;\\
		&t_\alpha=\left| {\dfrac{V_2-V_1}{\alpha}} \right|=\left| {\dfrac{2-0}{4}} \right|=0.5c;\\
		&t_b=\dfrac{g}{V}=\dfrac{10}{2}=5c;\\
		&t_c=\left| {\dfrac{V_2-V_1}{\alpha}} \right|=\left| {\dfrac{0-(-2)}{4}} \right|=0.5c;\\
	\end{split}				
\end{equation}

\subsubsection{Схема моделирования командного генератора:}
\begin{figure}[H]
	\begin{center}
		\includegraphics[scale=0.25]{g_2}
		\caption{- сигнал a(t)} 
		\label{pic:pic_1} % название для ссылок внутри кода
	\end{center}
\end{figure}

\subsubsection{Моделирование работы командного генератора:}
\begin{figure}[H]
	\begin{center}
		\includegraphics[scale=0.25]{g_3}
		\caption{- схема моделирования} 
		\label{pic:pic_1} % название для ссылок внутри кода
	\end{center}
\end{figure}

\begin{figure}[H]
	\begin{center}
		\includegraphics[scale=0.25]{g_4}
		\caption{- сигнал g(t)} 
		\label{pic:pic_1} % название для ссылок внутри кода
	\end{center}
\end{figure}

\newpage

\subsection{Исследование командного генератора возмущения}

\subsubsection{Моделирование работы командного генератора:}
\begin{figure}[H]
	\begin{center}
		\includegraphics[scale=0.3]{s_3}
		\caption{- сигнал V(t)} 
		\label{pic:pic_1} % название для ссылок внутри кода
	\end{center}
\end{figure}

\subsubsection{Моделирование работы командного генератора:}
\begin{figure}[H]
	\begin{center}
		\includegraphics[scale=0.25]{g_5}
		\caption{- сигнал g(t)} 
		\label{pic:pic_1} % название для ссылок внутри кода
	\end{center}
\end{figure}

\newpage

\section{Вывод}
В данной работе было проведено ознакомление с построением моделей внешних воздействий - сигналов задания и возмущений посредствам последовательного дифференцирования, который строит дифференциальное уравнение по известному частному решению, заданному в виде функции времени.
\end{document}
