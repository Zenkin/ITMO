\include{settings}

\begin{document}	% начало документа

% Титульная страница
\begin{titlepage}	% начало титульной страницы

	\begin{center}		% выравнивание по центру

		\large Санкт-Петербургский Национально Исследовательский Университет\\
		\large информационных технологий, механики и оптики \\
		\large Кафедра систем управления и информатики\\[6cm]
		% название института, затем отступ 6см
		
		\huge Технология изготовления элементов приборов и систем\\[0.5cm] % название работы, затем отступ 0,5см
		\large Отчет по лабораторной работе №5\\[0.1cm]
		\large Расчет режима резания при фрезеровании\\[1cm]
		\large Вариант №2\\[5cm]

	\end{center}


	\begin{flushright} % выравнивание по правому краю
		\begin{minipage}{0.25\textwidth} % врезка в половину ширины текста
			\begin{flushleft} % выровнять её содержимое по левому краю

				\large\textbf{Работу выполнили:}\\
				\large Зенкин А.М.\\
				\large Карпов К.В.\\
				\large {Группа:} P3335\\
				
				\large \textbf{Преподаватель:}\\
				\large Третьяков С.Д.

			\end{flushleft}
		\end{minipage}
	\end{flushright}
	
	\vfill % заполнить всё доступное ниже пространство

	\begin{center}
	\large Санкт-Петербург\\
	\large \the\year % вывести дату
	\end{center} % закончить выравнивание по центру

\thispagestyle{empty} % не нумеровать страницу
\end{titlepage} % конец титульной страницы

\vfill % заполнить всё доступное ниже пространство


% Содержание
\include{ToC}


\section{Цель работы}
Ознакомление с пакетом прикладных программ SIMULINK и основными приемами моделирования линейных динамических систем.


\section{Варианты параметров}

$n = 3$, $a_0 = 7$, $a_1 = 5$, $a_2 = 2$, $b_0 = 10$, $b_1 = 3$, $b_2 = 1.5$, $y(0) = 1$, $\dot{y}(0) = -0.5$, $\ddot{y}(0) = 0$\\

$n = 2$,  $A = \begin{bmatrix}
				0 & 1\\
				-5 & -0.5
				\end{bmatrix}$, $B = \begin{bmatrix}
									0.5\\
									1
									\end{bmatrix}$, $C^T = \begin{bmatrix}
															5\\
															0.5
															\end{bmatrix}$\\
															
$x_1(0) = 0.2$, $x_2(0) = -0.1$, $x_3(0) = -$


\section{Ход выполнения работы}

\subsection{Модель вход-выход}

\subsubsection{Математическая модель:}

\begin{equation}
	\begin{split}	
&\dddot{y} + 2\ddot{y} + 5\dot{y} + 7y = 1.5\ddot{u} + 3\dot{u} + 10u\\
&s^3y + 2s^2y + 5sy = 1.5s^2u + 3su + 10u | :s^3\\
&y = \frac{1}{s}(1.5u - 2y) + \frac{1}{s^2}(3u - 5y) + \frac{1}{s^3}(10u - 7y)\\
&z_1 = y\\
&z_1(0) = y(0) = 1\\
&\dot{y} = \dot{z_1} = \dot{z_2} + 1.5u - 2y\\
&z_2 = \dot{y} - 1.5u + 2y\\
&z_2(0) = \dot{y}(0) - 1.5u(0) + 2y(0) = -0.5 + 2*1 - 0 = 1.5\\
&z_2 = z_3 - 5y + 3u\\
&z_3 = /dot{z_2} + 5y -3u\\
&z_3 = \ddot{y} - 1.5\dot{u} + 2\dot{y} + 5\dot{y} -3\dot{u}\\
&z_3(0) = \ddot{y}(0) - 1.5\dot{u}(0) + 2\dot{y}(0) + 5\dot{y}(0) - 3\dot{u}(0) = 0 -0 - 1 - 5*0.5 - 0 = -3.5\\
	\end{split}
\end{equation}

\subsubsection{Моделирование системы при двух видах входного воздействия — u = 2sin(t)  и u = 1(t) — и нулевых начальных условиях:}

\begin{figure}[H]
	\begin{center}
		\includegraphics[scale=0.25]{SineWave}
		\caption{схема моделирования $u(t) = 2sin(t)$} 
		\label{pic:pic_1} % название для ссылок внутри кода
	\end{center}
\end{figure}

\begin{figure}[H]
	\begin{center}
		\includegraphics[scale=0.25]{SineWaveZeroCondition}
		\caption{графики сигналов $u(t)$ и $y(t)$} 
		\label{pic:pic_2} % название для ссылок внутри кода
	\end{center}
\end{figure}

\begin{figure}[H]
	\begin{center}
		\includegraphics[scale=0.25]{Const}
		\caption{схема моделирования $u(t) = 1$} 
		\label{pic:pic_3} % название для ссылок внутри кода
	\end{center}
\end{figure}

\begin{figure}[H]
	\begin{center}
		\includegraphics[scale=0.25]{ConstZeroCondition}
		\caption{графики сигналов $u(t)$ и $y(t)$} 
		\label{pic:pic_4} % название для ссылок внутри кода
	\end{center}
\end{figure}

\subsubsection{Моделирование свободного движения системы, т.е. с нулевым входным воздействием и ненулевыми начальными условиям:}

\begin{figure}[H]
	\begin{center}
		\includegraphics[scale=0.25]{ZeroImpactNoZeroConditionSimulink}
		\caption{схема моделирования с нулевым входным воздействием} 
		\label{pic:pic_5} % название для ссылок внутри кода
	\end{center}
\end{figure}

\begin{figure}[H]
	\begin{center}
		\includegraphics[scale=0.25]{ZeroImpactNoZeroCondition}
		\caption{график сигнала $y(t)$} 
		\label{pic:pic_6} % название для ссылок внутри кода
	\end{center}
\end{figure}

\newpage

\subsection{Модель вход-состояние-выход}
\subsubsection{Математическая модель:}

\begin{equation}
    \begin{matrix}
    \left\{
    \begin{matrix}
    \dot{x_1} = x_2 + 0.5u\\
    \dot{x_2} = -5x_1 - 0.5x_2 + u\\
    y = 5x_1 + 0.5x_2
    \end{matrix} \right.
    \end{matrix}
\end{equation}

\subsubsection{Моделирование системы при двух видах входного воздействия u = 2sin(t)  и u = 1(t) и нулевых начальных условиях:}

\begin{figure}[H]
	\begin{center}
		\includegraphics[scale=0.25]{SineWaveZeroConditionSim_2}
		\caption{схема моделирования $u(t) = 2sin(t)$} 
		\label{pic:pic_7} % название для ссылок внутри кода
	\end{center}
\end{figure}

\begin{figure}[H]
	\begin{center}
		\includegraphics[scale=0.25]{SineWaveZeroCOndition_2}
		\caption{графики сигналов $u(t)$ и $y(t)$} 
		\label{pic:pic_8} % название для ссылок внутри кода
	\end{center}
\end{figure}

\begin{figure}[H]
	\begin{center}
		\includegraphics[scale=0.25]{ConstZeroConditionSim_2}
		\caption{схема моделирования $u(t) = 1$} 
		\label{pic:pic_9} % название для ссылок внутри кода
	\end{center}
\end{figure}

\begin{figure}[H]
	\begin{center}
		\includegraphics[scale=0.25]{ConstZeroCondition_2}
		\caption{графики сигналов $u(t)$ и $y(t)$} 
		\label{pic:pic_10} % название для ссылок внутри кода
	\end{center}
\end{figure}

\subsubsection{Моделирование свободного движения системы, т.е. с нулевым входным воздействием и ненулевыми начальными условиям:}

\begin{figure}[H]
	\begin{center}
		\includegraphics[scale=0.25]{ZeroImpactNoZeroConditionSim_2}
		\caption{схема моделирования с нулевым входным воздействием} 
		\label{pic:pic_11} % название для ссылок внутри кода
	\end{center}
\end{figure}

\begin{figure}[H]
	\begin{center}
		\includegraphics[scale=0.25]{ZeroImpactNoZeroCondition_2}
		\caption{график сигнала $y(t)$}
		\label{pic:pic_12} % название для ссылок внутри кода
	\end{center}
\end{figure}


\section{Создание графиков}

\lstinputlisting[
	label=code:plot,
	caption={plot.m},% для печати символ '_' требует выходной символ '\'
]{plot.m}
\parindent=1cm % командна \lstinputlisting сбивает параментры отступа

\newpage

\section{Вывод}
В данной лабораторной работе было знакомство с базовыми функциями пакета прикладных программ для решения задач технических вычислений - Matlab. Также были приобретены навыки в работе с графической средой имитационного моделирования - Simulink. Было знакомство с основными приемами моделирования линейных динамических систем. Было проведено моделирование системы вход-выход и вход-состояние-выход при различных видах входного воздействия - u = 1 и u = 2sin(t) и нулевыми начальными условиями, а также свободного движения системы. Также были построены графики входных и выходных сигналов. При исследовании модели вход-состояния-выход все системы пришли к состояния равновесия. Рисунки \ref{pic:pic_8}, \ref{pic:pic_10}, \ref{pic:pic_12}. А при исследовании модели вход-выход с входным сигналом u = 2sin(t) получили незатухающие колебания, что говорит о нестабильности системы \ref{pic:pic_2}. При двух других случаях были получены модели, которые пришли к состоянию равновесия \ref{pic:pic_4} и \ref{pic:pic_6}.
\end{document}
