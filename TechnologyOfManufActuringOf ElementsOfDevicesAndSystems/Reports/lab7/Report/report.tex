\include{settings}

\begin{document}	% начало документа

% Титульная страница
\begin{titlepage}	% начало титульной страницы

	\begin{center}		% выравнивание по центру

		\large Санкт-Петербургский Национально Исследовательский Университет\\
		\large информационных технологий, механики и оптики \\
		\large Кафедра систем управления и информатики\\[6cm]
		% название института, затем отступ 6см
		
		\huge Технология изготовления элементов приборов и систем\\[0.5cm] % название работы, затем отступ 0,5см
		\large Отчет по лабораторной работе №5\\[0.1cm]
		\large Расчет режима резания при фрезеровании\\[1cm]
		\large Вариант №2\\[5cm]

	\end{center}


	\begin{flushright} % выравнивание по правому краю
		\begin{minipage}{0.25\textwidth} % врезка в половину ширины текста
			\begin{flushleft} % выровнять её содержимое по левому краю

				\large\textbf{Работу выполнили:}\\
				\large Зенкин А.М.\\
				\large Карпов К.В.\\
				\large {Группа:} P3335\\
				
				\large \textbf{Преподаватель:}\\
				\large Третьяков С.Д.

			\end{flushleft}
		\end{minipage}
	\end{flushright}
	
	\vfill % заполнить всё доступное ниже пространство

	\begin{center}
	\large Санкт-Петербург\\
	\large \the\year % вывести дату
	\end{center} % закончить выравнивание по центру

\thispagestyle{empty} % не нумеровать страницу
\end{titlepage} % конец титульной страницы

\vfill % заполнить всё доступное ниже пространство


% Содержание
\include{ToC}



\section{Цель работы}
Изучить методику расчета режима резания при шлифовании аналитическим способом. Приобрести навыки работы со справочной литературой.

\section{Варианты параметров}	
	
Материал заготовки и его свойства: Сталь 40Х незакаленная;\\

Вид обработки и параметр шерохоатости поверхности, мкм: Окончательная, $Ra=0,4$;\\

Размер шлифуемой поверхности, мм: $D=55h7$, $l=40$;\\

Припуск на сторону , мм: 0,15;\\

Кол-во одновре-менно обраба-тыва-емых деталей: 1;\\

Модель станка: 3М131;\\
				
\section{Ход выполнения работы}

\subsection{Описание:}
На круглошлифовальном станке 3М131 шлифуется шейка вала диаметром D=55h7 мм длиной l=40 мм, длина вала $l_1$=100 мм. Параметр шероховатости обработанной поверхности Ra=0,4 мкм. Припуск на сторону 0,15 мм. Материал заготовки – сталь 40X незакаленная, твердостью HB217.

\subsection{Выполнение эскиза обработки:}
\begin{figure}[H]
	\begin{center}
		\includegraphics[scale=0.6]{1}
		\caption{Эскиз обработки} 
		\label{pic:pic_1} % название для ссылок внутри кода
	\end{center}
\end{figure}

\subsection{Выбор шлифовального круга:}
Для фрезерования на вертикально-фрезерном станке заготовки из чугуна выбираем торцевую фрезу с пластинками из твердого сплава ВК6, диаметром D=(1,25$\div$1,5)$\cdot$В=($1,25\div$ 1,5)$\cdot$150=$187,5\div$225 мм. Принимаем D=200 мм; z=20, ГОСТ 9473-80.\\
   Геометрические параметры фрезы: $\phi$ =$60^{\circ}$, $\alpha=12^{\circ}$, $\gamma=10^{\circ}$, $\lambda=20^{\circ}$, $\phi_1=5^{\circ}$.\\
Схема установки фрезы – смещенная.

\subsection{Режим резания:}
\subsubsection{Глубина резания:}
Заданный припуск на чистовую обработку срезают за один проход, тогда:\\
\begin{equation}
	\begin{split}
		&t=h=4\;mm;
	\end{split}
\end{equation}

\subsubsection{Назначение подачи:}
Для получения шероховатости Ra=1,6 мкм подача на оборот $S_0$=1,1$\div$2,1 мм/об\\
\begin{equation}
	\begin{split}
		&S_z=\dfrac{S_0}{z} = \frac{2}{20}\ = 0,1\;mm/dent;
	\end{split}
\end{equation}

\subsubsection{Период стойкости:}
Для фрез торцевых диаметром от 200 мм до 250 с пластинками из твердого сплава применяют период стойкости:\\
\begin{equation}
	\begin{split}
		&T=240\;min;
	\end{split}
\end{equation}

\subsubsection{Скорость резания, допускаемая режущими свойствами инструмента:}
Для обработки серого чугуна фрезой диаметром от 200 до 250 мм, глубина резания t до 4 мм, подаче до 0,1 мм/зуб.:
\begin{equation}
	\begin{split}
		&V=148,38\;m/min;
	\end{split}
\end{equation}

С учетом поправочных коэффициентов:
\begin{equation}
	\begin{split}
		&K_{MV}=0,88,\; K_{NV}=1,\; K_{IV}=1;\\
		&V=V\cdot K_{MV}\cdot K_{NV}\cdot K_{IV}=1\cdot 1\cdot 148,38 = 130,57\; m/min;\\
	\end{split}
\end{equation}

Частота вращения шпинделя, соответствующая найденной скорости резания:
\begin{equation}
	\begin{split}
		&n=\dfrac{1000\cdot V}{\pi \cdot D}=\dfrac{1000\cdot 130,57}{3,14\cdot 200}=207,81\; rpm;\\
	\end{split}
\end{equation}

Корректируем по паспорту станка:
\begin{equation}
	\begin{split}
		&n=200\; rpm;\\
	\end{split}
\end{equation}

Действительная скорость резания
\begin{equation}
	\begin{split}
		&V_p=\frac{\pi \cdot D\cdot n}{1000}=\frac{3,14 \cdot 200\cdot 200}{1000}=125,6\; m/min;\\
	\end{split}
\end{equation}

\subsubsection{Минута подачи:}
\begin{equation}
	\begin{split}
		&S_M=S_z\cdot z\cdot n =0,1\cdot 20\cdot 200 = 400\;mm/min;\\
	\end{split}
\end{equation}
Это совпадает с паспортными данными станка.

\subsection{Мощность:}
\subsubsection{Мощность, затрачиваемая на резание:}
При фрезеровании чугуна с твердостью до НВ210, ширине фрезерования до 150 мм, глубине резания до 4 мм, подаче на зуб 0,1 мм/зуб, минутной подаче 400 мм/мин
\begin{equation}
	\begin{split}
		&N_p=11,75\;kW;\\
	\end{split}
\end{equation}

\subsubsection{Проверка достаточности мощности станка:}
Мощность на шпинделе станка: $N_{spindel}=N_d\cdot \eta;$

\begin{equation}
	\begin{split}
		&N_d=7,5\;k/W;\; \eta=0,8;\\
		&N_{sp} = 7,5\cdot 0,8 = 6\;kW;\\
	\end{split}
\end{equation}

Так как $N_{sp}=6 \;kW\; <\; N_p=11,75\;kW$, то обработка невозможна.\\

\subsection{Основное время:}
\begin{equation}
	\begin{split}
		&T_0=\frac{L}{S_M};\\
	\end{split}
\end{equation}
где $L=l+l_1;$\\

Расчёт основного времени не производится, так как мощность на шпинделе станка меньше требуемой мощности.\\

\section{Вывод}
В данной лабораторной работе была изучена методика расчёта режима резания при шлифовании аналитическим способом. Также были приобретены навыки работы со справочной литературой. Был построен эскиз обработки (рис. \ref{pic:pic_1}).
\end{document}

зучить методику расчета режима резания при шлифовании аналитическим способом. Приобрести навыки работы со справочной литературой.