\include{settings}

\begin{document}	% начало документа

% Титульная страница
\begin{titlepage}	% начало титульной страницы

	\begin{center}		% выравнивание по центру

		\large Санкт-Петербургский Национально Исследовательский Университет\\
		\large информационных технологий, механики и оптики \\
		\large Кафедра систем управления и информатики\\[6cm]
		% название института, затем отступ 6см
		
		\huge Технология изготовления элементов приборов и систем\\[0.5cm] % название работы, затем отступ 0,5см
		\large Отчет по лабораторной работе №5\\[0.1cm]
		\large Расчет режима резания при фрезеровании\\[1cm]
		\large Вариант №2\\[5cm]

	\end{center}


	\begin{flushright} % выравнивание по правому краю
		\begin{minipage}{0.25\textwidth} % врезка в половину ширины текста
			\begin{flushleft} % выровнять её содержимое по левому краю

				\large\textbf{Работу выполнили:}\\
				\large Зенкин А.М.\\
				\large Карпов К.В.\\
				\large {Группа:} P3335\\
				
				\large \textbf{Преподаватель:}\\
				\large Третьяков С.Д.

			\end{flushleft}
		\end{minipage}
	\end{flushright}
	
	\vfill % заполнить всё доступное ниже пространство

	\begin{center}
	\large Санкт-Петербург\\
	\large \the\year % вывести дату
	\end{center} % закончить выравнивание по центру

\thispagestyle{empty} % не нумеровать страницу
\end{titlepage} % конец титульной страницы

\vfill % заполнить всё доступное ниже пространство


% Содержание
\include{ToC}



\section{Цель работы}
Изучить методику расчета режима резания аналитическим способом. Ознакомиться и приобрести навыки работы со справочной литературой.

\section{Варианты параметров}			
Заготовка, материал и его свойства: Отливка с коркой. Серый чугун СЧ $20 HB160;$\\

Вид обработки и параметр шероховатости: Обтачивание на проход $Ra = 12,5$ мкм;\\

$D = 120 mm;$\\

$d = 110h12;$\\

$l = 310 mm;$\\									
\section{Ход выполнения работы}

\subsection{Описание:}
На токарно-винторезном станке 16К20 производится черновое обтачивание на проход вала $D = 120$ мм до $d = 110h12$ мм. Длина обрабатываемой поверхности 310 мм; длина вала $l_1= 430$ мм. Заготовка - серый чугун 20 СЧ с пределом прочности $\sigma_b = 200$ МПа. Способ крепления заготовки - в центрах и поводковом патроне. Система СПИД недостаточно жесткая. Параметр шероховатости поверхности $Ra = 12,5$ мкм. Необходимо: выбрать режущий инструмент, назначить режим резания; определить основное время.

\subsection{Выполнение эскиза обработки:}
\begin{figure}[H]
	\begin{center}
		\includegraphics[scale=0.7]{1}
		\caption{Эскиз обработки} 
		\label{pic:pic_1} % название для ссылок внутри кода
	\end{center}
\end{figure}

\subsection{Выбор режущего инструмента:}
Для обтачивания на проход вала из стали 40Х принимаем токарный проходной резец прямой правый с пластинкой из твердого сплава ВК6. Форма передней поверхности радиусная с фаской; геометрические параметры режущей части резца:\\
$\gamma=15^{\circ}$; $\alpha=12^{\circ}$; $\lambda=0$,\\ 
$\phi=60^{\circ}$ ; $\phi_1=30^{\circ}$,\\
$r=1$ мм; $f=1$ мм.

\subsection{Назначение глубины резания:}
Глубина резания. При черновой обработке припуск срезаем за один проход, тогда:\\
\begin{equation}
	\begin{split}
		&t=h=\dfrac{D-d}{2}=\dfrac{120-110}{2} = 5 mm;
	\end{split}
\end{equation}

\subsection{Определение подачи:}
Назначаем подачу. Для черновой обработки заготовки из серого чегуна диаметром до 400 мм резцом сечения $16$x$25$ при глубине резания до 5мм. $S=0,9$ мм/об.

\subsection{Рассчёт скорости резания:}
Скорость резания , допускаемая материалом резца:
\begin{equation}
	\begin{split}
		&V=\dfrac{C_v}{T^mt^xS^y}K_v, m/min\\
		&C_v=243, x=0.15, y=0.4, m=0.2, T=60 min\\
	\end{split}
\end{equation}
Поправочный коэффициент для обработки резцом с твердосплавной пластиной:
\begin{equation}
	\begin{split}
		&K_v = K_{mv}\cdot K_{nv}\cdot K_{uv}\cdot K_{\phi v};\\
		&K_{mv} = \left( \dfrac{190}{\sigma_b} \right)^{n_v}, n_v = 1;\\
		&K_{nv}=0.8, \\
		&K_{uv}=0.83, \\
		&K_{\phi v}=0.9; \\
		&V = \dfrac{243}{60^{0.2}\cdot 5^{0.15}\cdot 0.9^{0.4}}\cdot 1.34 \cdot 0.83\cdot 0.8\cdot 0.9 = 73.65 m/min
	\end{split}
\end{equation}

\newpage

\subsection{Определение частоты вращения шпинделя и корректирование по паспорту станка:}
Частота вращения, соответствующая найденной скорости резания:
\begin{equation}
	\begin{split}
		&n=\dfrac{1000V}{\pi \cdot D}, rpm;\\
		&n=\dfrac{1000\cdot 73.65}{3.14\cdot 120}=195.47 rpm;\\
	\end{split}
\end{equation}

Корректируем частоту вращения шпинделя по паспортным данным станка:
\begin{equation}
	\begin{split}
		&n_d = 200 rpm;\\
	\end{split}
\end{equation}

\subsection{Определение действительной скорости резания:}
Действительная скорость резания:
\begin{equation}
	\begin{split}
		&V_d = \dfrac{\pi\cdot D\cdot n}{1000}, m/min;\\
		&V_d = \dfrac{3.14\cdot 120\cdot 200}{1000} =75.36 m/min;
	\end{split}
\end{equation}

\subsection{Рассчёт основного технологического времени:}
Основное время:
\begin{equation}
	\begin{split}
		&T_o = \dfrac{L}{n\cdot S}\cdot i, min;
	\end{split}
\end{equation}
Путь резца: $L=1+y+\delta$,  мм;\\
Врезание резца: $y=t\cdot ctg(\phi) = 5\cdot ctg(60^{\circ}) = 5\cdot 0.58 = 2.9$ мм;\\
Пробег резца: $\delta = 2.5$ мм;\\
Тогда: $L=310 + 2.5 + 2.9 = 315.4$ мм;\\
\begin{equation}
	\begin{split}
		&T_o = \dfrac{315.4}{200\cdot 0.9}\cdot = 1.75 min;
	\end{split}
\end{equation}

\section{Вывод}
В данной лабораторной работе была изучена методика расчёта режима резания аналитическим способом, а также были приобретены навыки работы со справочной литературой. Был построен эскиз обработки (рис. \ref{pic:pic_1}).
\end{document}
