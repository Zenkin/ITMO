\include{settings}

\begin{document}	% начало документа

% Титульная страница
\begin{titlepage}	% начало титульной страницы

	\begin{center}		% выравнивание по центру

		\large Санкт-Петербургский Национально Исследовательский Университет\\
		\large информационных технологий, механики и оптики \\
		\large Кафедра систем управления и информатики\\[6cm]
		% название института, затем отступ 6см
		
		\huge Технология изготовления элементов приборов и систем\\[0.5cm] % название работы, затем отступ 0,5см
		\large Отчет по лабораторной работе №5\\[0.1cm]
		\large Расчет режима резания при фрезеровании\\[1cm]
		\large Вариант №2\\[5cm]

	\end{center}


	\begin{flushright} % выравнивание по правому краю
		\begin{minipage}{0.25\textwidth} % врезка в половину ширины текста
			\begin{flushleft} % выровнять её содержимое по левому краю

				\large\textbf{Работу выполнили:}\\
				\large Зенкин А.М.\\
				\large Карпов К.В.\\
				\large {Группа:} P3335\\
				
				\large \textbf{Преподаватель:}\\
				\large Третьяков С.Д.

			\end{flushleft}
		\end{minipage}
	\end{flushright}
	
	\vfill % заполнить всё доступное ниже пространство

	\begin{center}
	\large Санкт-Петербург\\
	\large \the\year % вывести дату
	\end{center} % закончить выравнивание по центру

\thispagestyle{empty} % не нумеровать страницу
\end{titlepage} % конец титульной страницы

\vfill % заполнить всё доступное ниже пространство


% Содержание
\include{ToC}


\section{Цель работы}
Выборать двигатель для САУ.


\section{Варианты параметров}

\begin{center}
	$K_1=0.15\text{ A/B}$;\\
	$T_1=0.7\text{ мс}$;\\
	$K_M=0.12\text{ Нм/A}$;\\
	$K_E=0.1\text{ Вс}$;\\
	$J=2.7\cdot 10^{-3}\text{кг}\cdot\text{м}^2$;\\
	$M_H=0.16\text{ Нм}$;\\
\end{center}

\section{Ход выполнения работы}
\subsection{Модель БДПТ, построенная на основе двухфазной синхронной электрической машины}
\subsubsection{Схема моделирование двухфазной электрической машины:}
Схема моделирование двухфазной электрической машины. \ref{pic:pic_1}.
\begin{figure}[H]
	\begin{center}
		\includegraphics[scale=0.3]{1}
		\caption{- схема моделирование} 
		\label{pic:pic_1} % название для ссылок внутри кода
	\end{center}
\end{figure}

\newpage

\subsubsection{График переходного процесса двигателя по скорости:}
График переходного процесса двигателя по скорости представлен на рисунке \ref{pic:pic_2}.
\begin{figure}[H]
	\begin{center}
		\includegraphics[scale=0.3]{2}
		\caption{- схема моделирование} 
		\label{pic:pic_2} % название для ссылок внутри кода
	\end{center}
\end{figure}

\subsubsection{Определение вида и параметров математической модели вход-выход:}

Апериодическое звено 1-го порядка описывается дифференциальным уравнением: 
\begin{equation}
	\begin{split}
		&T\cdot \dot{y}+y=k\cdot g;\\
	\end{split}
\end{equation}
или в виде передаточной функции:
\begin{equation}
	\begin{split}
		&y=\dfrac{k}{Ts+1};\\
	\end{split}
\end{equation}
, где
\begin{center}
	$k=138.4, T=0.78$;
\end{center}
Тогда передаточная функция примет вид:
\begin{equation}
	\begin{split}
		&y=\dfrac{138.4}{0.78s+1};\\
	\end{split}
\end{equation}

\newpage

\subsubsection{Фазовая и временная диаграммы токов при работе двигателя в номинальном режиме:}
Фазовая и временная диаграммы изображены на рисунках \ref{pic:pic_3} и \ref{pic:pic_4}.
\begin{figure}[H]
	\begin{center}
		\includegraphics[scale=0.3]{faz}
		\caption{- фазовая диаграмма} 
		\label{pic:pic_3} % название для ссылок внутри кода
	\end{center}
\end{figure}

\begin{figure}[H]
	\begin{center}
		\includegraphics[scale=0.3]{tok}
		\caption{- временная диаграмма} 
		\label{pic:pic_4} % название для ссылок внутри кода
	\end{center}
\end{figure}

\newpage

\subsubsection{Механическая и регулировочная характеристики двигателя:}
Механическая и регулировочная характеристики приведены на рисунках \ref{pic:pic_5} и \ref{pic:pic_6}.
\begin{figure}[H]
	\begin{center}
		\includegraphics[scale=0.28]{mex}
		\caption{- механическая характеристика} 
		\label{pic:pic_5} % название для ссылок внутри кода
	\end{center}
\end{figure}

\begin{figure}[H]
	\begin{center}
		\includegraphics[scale=0.28]{reg}
		\caption{- регулировочная характеристика} 
		\label{pic:pic_6} % название для ссылок внутри кода
	\end{center}
\end{figure}
\subsection{Модель БДПТ, построенная на основе двухфазной синхронной электрической машины}

\subsubsection{График переходного процесса двигателя по скорости:}
График переходногопроцесса и временная диаграмма токов приведены на рисунках \ref{pic:pic_7} и \ref{pic:pic_8}.
\begin{figure}[H]
	\begin{center}
		\includegraphics[scale=0.23]{omega_2}
		\caption{- график переходного процесса по скорости} 
		\label{pic:pic_7} % название для ссылок внутри кода
	\end{center}
\end{figure}

\subsubsection{Временная диаграмма токов при работе двигателя в номинальном режиме:}

\begin{figure}[H]
	\begin{center}
		\includegraphics[scale=0.25]{tok_2}
		\caption{- временная диаграмма токов} 
		\label{pic:pic_8} % название для ссылок внутри кода
	\end{center}
\end{figure}
\newpage

\section{Вывод}
В данной лабораторной работе были рассмотрены модели двухфазной и трехфазной синхронной электрической машины. Были получены графики переходных процессов по скорости, механическая и регулировочная характиристики для двухфазной синхронной электрической машины и фазовые и времннные диаграммы при номинальном режиме работы двигателя. Характеристики линейны и практически совпадают с характеристиками реальной синхронной машины. Фазные токи имеют синусоидальную форму.
\end{document}
