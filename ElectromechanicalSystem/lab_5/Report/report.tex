\include{settings}

\begin{document}	% начало документа

% Титульная страница
\begin{titlepage}	% начало титульной страницы

	\begin{center}		% выравнивание по центру

		\large Санкт-Петербургский Национально Исследовательский Университет\\
		\large информационных технологий, механики и оптики \\
		\large Кафедра систем управления и информатики\\[6cm]
		% название института, затем отступ 6см
		
		\huge Технология изготовления элементов приборов и систем\\[0.5cm] % название работы, затем отступ 0,5см
		\large Отчет по лабораторной работе №5\\[0.1cm]
		\large Расчет режима резания при фрезеровании\\[1cm]
		\large Вариант №2\\[5cm]

	\end{center}


	\begin{flushright} % выравнивание по правому краю
		\begin{minipage}{0.25\textwidth} % врезка в половину ширины текста
			\begin{flushleft} % выровнять её содержимое по левому краю

				\large\textbf{Работу выполнили:}\\
				\large Зенкин А.М.\\
				\large Карпов К.В.\\
				\large {Группа:} P3335\\
				
				\large \textbf{Преподаватель:}\\
				\large Третьяков С.Д.

			\end{flushleft}
		\end{minipage}
	\end{flushright}
	
	\vfill % заполнить всё доступное ниже пространство

	\begin{center}
	\large Санкт-Петербург\\
	\large \the\year % вывести дату
	\end{center} % закончить выравнивание по центру

\thispagestyle{empty} % не нумеровать страницу
\end{titlepage} % конец титульной страницы

\vfill % заполнить всё доступное ниже пространство


% Содержание
\include{ToC}


\section{Цель работы}
Исследование принципов построения многоконтурных электромеханических систем автоматического управления.

\section{Ход выполнения работы}

\subsection{Выбраем структуру регулятора тока и на основе стандартных методов настройки рассчитываем необходимые значения параметров закона управления}

\subsubsection{Схема моделирования}
Схема моделирования представлена на рисунке \ref{pic:pic_1}.
\begin{figure}[H]
	\begin{center}
		\includegraphics[scale=0.3]{1}
		\caption{- схема моделирование} 
		\label{pic:pic_1} % название для ссылок внутри кода
	\end{center}
\end{figure}

\subsubsection{Передаточная функция регулятора по току}
\begin{center}
	$W_{pm}(s)=\dfrac{1}{T_u\cdot s};$\\
	$W_{pm}(s)=\dfrac{1}{0.0321\cdot s};$\\
\end{center}

\newpage

\subsubsection{Переходная характеристика и ее параметры}
Схема переходного процесса представлена на рисунке \ref{pic:pic_2}.
\begin{figure}[H]
	\begin{center}
		\includegraphics[scale=0.2]{1_1}
		\caption{- переходный процесс по току} 
		\label{pic:pic_2} % название для ссылок внутри кода
	\end{center}
\end{figure}

Параметры:
\begin{center}
	$T = 0.1 \text{ с}$;\\
	$\text{Перерегулирование} \; 13.2\%$;\\
\end{center}

\subsection{Выбраем структуру регулятора скорости и на основе стандартных методов настройки рассчитываем необходимые значения параметров закона управления}

\subsubsection{Схема моделирования}
Схема моделирования представлена на рисунке \ref{pic:pic_3}.
\begin{figure}[H]
	\begin{center}
		\includegraphics[scale=0.4]{2}
		\caption{- схема моделирования} 
		\label{pic:pic_3} % название для ссылок внутри кода
	\end{center}
\end{figure}

\subsubsection{Передаточная функция регулятора по скорости}
\begin{center}
	$W_{pm}(s)=\dfrac{K_{pc}\cdot(T_{uc}\cdot s+1)}{T_{uc}\cdot s};$\\
	$W_{pm}(s)=\dfrac{0.0182\cdot s+0.1723}{0.072\cdot s};$\\
\end{center}


\subsubsection{Переходная характеристика и ее параметры}
Схема переходного процесса представлена на рисунке \ref{pic:pic_4}.
\begin{figure}[H]
	\begin{center}
		\includegraphics[scale=0.2]{2_1}
		\caption{- переходный процесс по скорости} 
		\label{pic:pic_4} % название для ссылок внутри кода
	\end{center}
\end{figure}

Параметры:
\begin{center}
	$T = 0.39 \text{ с}$;\\
\end{center}

\newpage

\subsection{Выбраем структуру регулятора положения и на основе стандартных методов настройки рассчитываем необходимые значения параметров закона управления}

\subsubsection{Схема моделирования}
Схема моделирования представлена на рисунке \ref{pic:pic_5}.
\begin{figure}[H]
	\begin{center}
		\includegraphics[scale=0.3]{3_1}
		\caption{- схема моделирование} 
		\label{pic:pic_5} % название для ссылок внутри кода
	\end{center}
\end{figure}

\subsubsection{Передаточная функция регулятора по положению}
\begin{center}
	$W_{p\text{п}}(s)=\dfrac{1}{4\cdot T_c \cdot K_{03}};$\\
	$W_{p\text{п}}(s)=70.78;$\\
\end{center}

\newpage

\subsubsection{Переходная характеристика и ее параметры}
Схема переходного процесса представлена на рисунке \ref{pic:pic_6}.
\begin{figure}[H]
	\begin{center}
		\includegraphics[scale=0.2]{3}
		\caption{- переходный процесс по току} 
		\label{pic:pic_6} % название для ссылок внутри кода
	\end{center}
\end{figure}

Параметры:
\begin{center}
	$T = 2.8 \text{ с}$;\\
\end{center}

\section{Вывод}
В данной лабораторной работе были рассмотрены принципы построения многоконтурных электромеханических систем автоматического управления. Также были расчитаны все значения регуляторов, построены переходные процессы и определены их параметры.
\end{document}