\include{settings}

\begin{document}	% начало документа

% Титульная страница
\begin{titlepage}	% начало титульной страницы

	\begin{center}		% выравнивание по центру

		\large Санкт-Петербургский Национально Исследовательский Университет\\
		\large информационных технологий, механики и оптики \\
		\large Кафедра систем управления и информатики\\[6cm]
		% название института, затем отступ 6см
		
		\huge Технология изготовления элементов приборов и систем\\[0.5cm] % название работы, затем отступ 0,5см
		\large Отчет по лабораторной работе №5\\[0.1cm]
		\large Расчет режима резания при фрезеровании\\[1cm]
		\large Вариант №2\\[5cm]

	\end{center}


	\begin{flushright} % выравнивание по правому краю
		\begin{minipage}{0.25\textwidth} % врезка в половину ширины текста
			\begin{flushleft} % выровнять её содержимое по левому краю

				\large\textbf{Работу выполнили:}\\
				\large Зенкин А.М.\\
				\large Карпов К.В.\\
				\large {Группа:} P3335\\
				
				\large \textbf{Преподаватель:}\\
				\large Третьяков С.Д.

			\end{flushleft}
		\end{minipage}
	\end{flushright}
	
	\vfill % заполнить всё доступное ниже пространство

	\begin{center}
	\large Санкт-Петербург\\
	\large \the\year % вывести дату
	\end{center} % закончить выравнивание по центру

\thispagestyle{empty} % не нумеровать страницу
\end{titlepage} % конец титульной страницы

\vfill % заполнить всё доступное ниже пространство


% Содержание
\include{ToC}


\section{Типы заземляющих устройств}
Для заземления электроустановок используется заземляющее устройство (рис. 3.8), состоящее из заземлителя – одиночного или группы заземлителей 1, конструктивно объединенных соединительной полосой 2, размещаемых в земле, и заземляющего проводника 3, соединяющего заземляемую установку 4 через проложенную по стенам помещения магистраль заземления 5 с заземлителем.

Для присоединения заземляющего проводника на корпусе установки должен быть предусмотрен элемент для заземления – болт 6 (винт, шпилька), выполненный из металла, стойкого к коррозии. Возле болта должен быть нанесен нестираемый при эксплуатации знак заземления 7.


\begin{figure}[H]
	\begin{center}
		\includegraphics[scale=1.4]{pic_1.jpg}
		\caption{ - напряжение шага при одиночном и групповом заземлителе} 
		\label{pic:pic_1} % название для ссылок внутри кода
	\end{center}
\end{figure}

В качестве искусственных заземлителей, располагаемых в земле чаще всего вертикально, используются стальные трубы диаметром 5 – 6 см с толщиной стенки не менее 3,5 мм или уголки с толщиной полок не менее 4 мм и размерами от 40x40 мм до 60x60 мм длиной 2,5 – 3 м.

Различают два типа заземляющих устройств: выносное и контурное.

В выносном заземляющем устройстве заземлитель вынесен за пределы площадки, на которой размещено заземляемое электрооборудование. Такой тип заземляющего устройства применяют только при малых значениях тока замыкания на землю.

В контурном заземляющем устройстве одиночные вертикальные зазем-лители располагают по контуру (периметру) площадки, на которой нахо-дится заземляемое оборудование (см. рис. 3.9), или заземлители распреде-ляют по всей площадке по возможности равномерно.

\begin{figure}[H]
	\begin{center}
		\includegraphics[scale=1.4]{pic_2.jpg}
		\caption{ - устройство защитного заземления} 
		\label{pic:pic_1} % название для ссылок внутри кода
	\end{center}
\end{figure}

1 – электроустановка; 2 – заземляющий проводник; 3 – магистраль заземления; 4 – перемычка; 5 –заземлитель; 6 – соединительная полоса
\\

Безопасность при использовании контурного заземляющего устройства обеспечивается не только уменьшением потенциала заземленного оборудо-вания, но и выравниванием и повышением потенциала на поверхности за-щищаемой территории путем размещения одиночных заземлителей на определенном расстоянии друг от друга (менее 40 м).

Расчёт контурного заземляющего устройства сводится к определению количества заземлителей, длины соединительной полосы и схемы размещения заземлителей в земле на защищаемой территории, при которых сопротивление заземляющего устройства и напряжение прикосновения не превысят допустимых значений.

\section{Входные данные}

№ варианта: 3\\
Вид заземлителя: Труба; Вид соедин.полосы: Полоса;\\
h = 0, м; $l_0$ = 2.9, м; $d_\text{тр}$ = 40, мм; b = 40, мм; a = 4.5, м;\\
$k_\text{сез}$ = 1.3; $\eta_\text{з}$ = 0.65; $\eta_\text{п}$ = 0.3; $\rho$ = 70, Ом*м \\ 

\section{Расчёт защитного заземления}

Произвести расчёт параметров контурного
заземляющего устройства для защитного заземления электроустановок по
следующим исходным данным: заземлители – стальные трубы диаметром
50 мм, длиной 3 м забиваются в землю на глубину 0,6 м от ее поверхности;
соединительная полоса – стальная полоса шириной 40 мм; грунт – глина,
удельное сопротивление которой ρ = 60 Ом·м. Расстояние между двумя
соседними заземлителями 4 м; значения коэффициентов: к сез = 1,4; η з = 0,7;
η п = 0,4.

\subsection{Определяем сопротивление одного вертикального заземлителя растеканию тока в земле:}

\begin{equation}
    \begin{split}	
        &R_\text{pipe}=\frac{\rho_p}{2\pi \cdot l_0}\cdot ln \frac{4 l_0}{d_{pipe}} = \frac{1.3 \cdot 70}{2\pi \cdot 2.9}\cdot ln \frac{4 \cdot 2.9}{40} = 28.331\; Om;\\
    \end{split}
\end{equation}

\subsection{Определяем необходимое число вертикальных заземлителей n:}

\begin{equation}
    \begin{split}	
        &n=\frac{R_0}{R_{z} \cdot \eta_{z}} = \frac{28.331}{4 \cdot 0.65} = 10.896\; pcs;\\
    \end{split}
\end{equation}

Полученное значение следует округлять в меньшую сторону до целого числа. Принимаем число заземлителей n = 10 шт.;

\subsection{Рассчитываем длину соединительной полосы l:}

\begin{equation}
    \begin{split}	
        &l = 1.05\cdot an = 1.05\cdot 4.5\cdot 10 = 47.25\; м;\\
    \end{split}
\end{equation}

\subsection{Определяем сопротивление соединительной полосы растеканию тока:}

\begin{equation}
    \begin{split}	
        &R_{p} = \frac{\rho_p}{2 \pi \cdot l} ln \frac{2 l }{0.5 \cdot b} =\frac{1.3 \cdot 70}{2 \pi \cdot 47.25} ln \frac{2\cdot 47.25 }{0.5 \cdot 40} = 0,476\; Om;\\
    \end{split}
\end{equation}

\subsection{Рассчитываем полное сопротивление заземляющего устройства:}

\begin{equation}
    \begin{split}	
        &R_{zy} = \frac{R_{pipe}\cdot R_p}{R_z \eta_{p} + nR_p\cdot \eta_{z}} = \frac{28.331\cdot 0.476}{4\cdot 0.3 + 10\cdot 0.476p\cdot 0.65} = 1.164\; Om;\\
    \end{split}
\end{equation}

Расчёт считаем законченным, так как сопротивление проектируемого заземляющего устройства менее 4 Ом, что соответствует требованиям ПУЭ.



\end{document}
