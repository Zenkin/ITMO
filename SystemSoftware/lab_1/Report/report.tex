\include{settings}

\begin{document}	% начало документа

% Титульная страница
\begin{titlepage}	% начало титульной страницы

	\begin{center}		% выравнивание по центру

		\large Санкт-Петербургский Национально Исследовательский Университет\\
		\large информационных технологий, механики и оптики \\
		\large Кафедра систем управления и информатики\\[6cm]
		% название института, затем отступ 6см
		
		\huge Технология изготовления элементов приборов и систем\\[0.5cm] % название работы, затем отступ 0,5см
		\large Отчет по лабораторной работе №5\\[0.1cm]
		\large Расчет режима резания при фрезеровании\\[1cm]
		\large Вариант №2\\[5cm]

	\end{center}


	\begin{flushright} % выравнивание по правому краю
		\begin{minipage}{0.25\textwidth} % врезка в половину ширины текста
			\begin{flushleft} % выровнять её содержимое по левому краю

				\large\textbf{Работу выполнили:}\\
				\large Зенкин А.М.\\
				\large Карпов К.В.\\
				\large {Группа:} P3335\\
				
				\large \textbf{Преподаватель:}\\
				\large Третьяков С.Д.

			\end{flushleft}
		\end{minipage}
	\end{flushright}
	
	\vfill % заполнить всё доступное ниже пространство

	\begin{center}
	\large Санкт-Петербург\\
	\large \the\year % вывести дату
	\end{center} % закончить выравнивание по центру

\thispagestyle{empty} % не нумеровать страницу
\end{titlepage} % конец титульной страницы

\vfill % заполнить всё доступное ниже пространство


% Содержание
\include{ToC}


\section{Цель работы}
Изучение команд для получения информации о системе.Получить навыки работы с каталогами, папками и файлами в ОС Linux Ubuntu. 


\section{Задание 1.1 Средства просмотра системной информации}

\subsection{uname - консольная UNIX-утилита, выводящая информацию о системе.}

\begin{figure}[H]
	\begin{center}
		\includegraphics[scale=0.5]{type_of_processor}
		\caption{Тип процессора} 
		\label{pic:pic_1} % название для ссылок внутри кода
	\end{center}
\end{figure}

\newpage

\subsection{date — утилита Unix для работы с системными часами. Выводит текущую дату и время в различных форматах и позволяет устанавливать системное время.}

\begin{figure}[H]
	\begin{center}
		\includegraphics[scale=0.5]{time_1}
		\caption{Текущее время 1} 
		\label{pic:pic_2} % название для ссылок внутри кода
	\end{center}
\end{figure}

\begin{figure}[H]
	\begin{center}
		\includegraphics[scale=0.5]{time_2}
		\caption{Текущее время 2} 
		\label{pic:pic_3} % название для ссылок внутри кода
	\end{center}
\end{figure}

\newpage


\section{Задание 1.2 Команды для работы с каталогами, папками и файлами}

\subsection{Отображаем текущее положение (путь к директории, в которой мы сейчас находимся).}

\subsubsection{pwd}
Консольная утилита в UNIX-подобных системах, которая выводит полный путь от корневого каталога к текущему рабочему каталогу: в контексте которого (по умолчанию) будут исполняться вводимые команды.

\begin{figure}[H]
	\begin{center}
		\includegraphics[scale=0.5]{1}
		\caption{} 
		\label{pic:pic_4} % название для ссылок внутри кода
	\end{center}
\end{figure}

\newpage

\subsection{Создали директорию 1. Подтвердили создание папки (вывели содержимое директории). Перешли в эту папку.}

\subsubsection{cd}
Команда cd позволяет перемещаться из одного каталога в другой.
\subsubsection{mkdir}
Команда для создания новых каталогов.
\subsubsection{ls}
Команда, которая печатает в стандартный вывод содержимое каталогов.

\begin{figure}[H]
	\begin{center}
		\includegraphics[scale=0.5]{2}
		\caption{} 
		\label{pic:pic_5} % название для ссылок внутри кода
	\end{center}
\end{figure}

\subsection{Создали одновременно в папке 1 папки Zenkin, Karpov. Подтвердили создание папки.}

\begin{figure}[H]
	\begin{center}
		\includegraphics[scale=0.5]{3}
		\caption{} 
		\label{pic:pic_6} % название для ссылок внутри кода
	\end{center}
\end{figure}

\subsection{Не переходя в папку Zenkin создали папку ArtemiiKarpovKonstantin.  Подтвердить создание папки. }

\begin{figure}[H]
	\begin{center}
		\includegraphics[scale=0.5]{4}
		\caption{} 
		\label{pic:pic_7} % название для ссылок внутри кода
	\end{center}
\end{figure}

\subsection{Перешли в папку ArtemiiKarpovKonstantin(используя относительный путь). Одновременно создали два файла Artemii и Kirill. Подтвердили создание файлов.}

\subsubsection{touch}
Команда Unix, предназначенная для установки времени последнего изменения файла или доступа в текущее время. Также используется для создания пустых файлов.

\begin{figure}[H]
	\begin{center}
		\includegraphics[scale=0.5]{5}
		\caption{} 
		\label{pic:pic_8} % название для ссылок внутри кода
	\end{center}
\end{figure}

\newpage

\subsection{Создали file4, не переходя в. Подтвердили создание файла.}

\begin{figure}[H]
	\begin{center}
		\includegraphics[scale=0.5]{6}
		\caption{} 
		\label{pic:pic_9} % название для ссылок внутри кода
	\end{center}
\end{figure}

\subsection{Скопировали папку ArtemiiKirillKonstantin в папку 1 со всем содержимым. Подтвердили.}

\subsubsection{cp}
Команда Unix в составе GNU Coreutils, предназначенная для копирования файлов из одного в другие каталоги (возможно, с другой файловой системой). Исходный файл остаётся неизменным, имя созданного файла может быть таким же, как у исходного, или изменится.

\begin{figure}[H]
	\begin{center}
		\includegraphics[scale=0.5]{7}
		\caption{} 
		\label{pic:pic_10} % название для ссылок внутри кода
	\end{center}
\end{figure}

\subsection{Из папки ArtemiiKirillKonstantin скопировали файл с именем, соответствующим фамилии Zenkin в папку Zenkin. Удалили папку ArtemiiKirillKonstantin со всем содержимым. Подтвердили.}

\subsubsection{rm}
Утилита в UNIX и UNIX-подобных системах, используемая для удаления файлов из файловой системы.

\begin{figure}[H]
	\begin{center}
		\includegraphics[scale=0.5]{8}
		\caption{} 
		\label{pic:pic_11} % название для ссылок внутри кода
	\end{center}
\end{figure}

\newpage

\subsection{Одновременно переименовали file4 в «Пустой» и переместили его в папку 1. Подтвердили.}

\subsubsection{mv}
Утилита в UNIX и UNIX-подобных системах, используется для перемещения или переименования файлов.

\begin{figure}[H]
	\begin{center}
		\includegraphics[scale=0.5]{9}
		\caption{} 
		\label{pic:pic_12} % название для ссылок внутри кода
	\end{center}
\end{figure}

\newpage

\subsection{Из папки ArtemiiKirillKonstantin переместили файлы с именами в Karpov. Подтвердили. Оставили в папке файл, соответствующий фамилии, другой файл удалили.  Подтвердили. Удалили пустую папку ArtemiiKirillKonstantin (используя команду для удаления пустой папки). Просмотрели содержимое папок 1, Zenkin, Karpov.}

\subsubsection{find}
Утилита поиска файлов по имени и другим свойствам, используемая в UNIX-подобных операционных системах.
\begin{figure}[H]
	\begin{center}
		\includegraphics[scale=0.5]{10}
		\caption{} 
		\label{pic:pic_13} % название для ссылок внутри кода
	\end{center}
\end{figure}

\begin{figure}[H]
	\begin{center}
		\includegraphics[scale=0.44]{11}
		\caption{} 
		\label{pic:pic_14} % название для ссылок внутри кода
	\end{center}
\end{figure}

\subsection{Перешли из текущей папки в папку 1, используя специальные символы. Создали текстовый документ file1 (запустив консольный текстовый редактор) и записали в него текст: «Все задания выполнили. Команды для работы с папками, файлами и каталогами выучили». Сохранили файл под названием finita. Сделали скрин текстового редактора с введенным текстом. Подтвердили наличие файла.}

\subsubsection{nano}
Консольный текстовый редактор для UNIX.
\begin{figure}[H]
	\begin{center}
		\includegraphics[scale=0.5]{12}
		\caption{} 
		\label{pic:pic_15} % название для ссылок внутри кода
	\end{center}
\end{figure} 

\begin{figure}[H]
	\begin{center}
		\includegraphics[scale=0.5]{13}
		\caption{} 
		\label{pic:pic_16} % название для ссылок внутри кода
	\end{center}
\end{figure}

\subsection{Вывели содержание файла finite в терминале.}

\subsubsection{cat}
утилита UNIX, выводящая последовательно указанные файлы (или устройства), таким образом, объединяя их в единый поток.

\begin{figure}[H]
	\begin{center}
		\includegraphics[scale=0.5]{14}
		\caption{} 
		\label{pic:pic_18} % название для ссылок внутри кода
	\end{center}
\end{figure}

\section{Вывод}
В данно лабораторной работе были изучены команды для получения информации о системе, а также навыки работы с каталогами, папками и файлами в ОС Linux Ubuntu. 
\end{document}
