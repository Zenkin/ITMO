\include{settings}

\begin{document}	% начало документа

% Титульная страница
\begin{titlepage}	% начало титульной страницы

	\begin{center}		% выравнивание по центру

		\large Санкт-Петербургский Национально Исследовательский Университет\\
		\large информационных технологий, механики и оптики \\
		\large Кафедра систем управления и информатики\\[6cm]
		% название института, затем отступ 6см
		
		\huge Технология изготовления элементов приборов и систем\\[0.5cm] % название работы, затем отступ 0,5см
		\large Отчет по лабораторной работе №5\\[0.1cm]
		\large Расчет режима резания при фрезеровании\\[1cm]
		\large Вариант №2\\[5cm]

	\end{center}


	\begin{flushright} % выравнивание по правому краю
		\begin{minipage}{0.25\textwidth} % врезка в половину ширины текста
			\begin{flushleft} % выровнять её содержимое по левому краю

				\large\textbf{Работу выполнили:}\\
				\large Зенкин А.М.\\
				\large Карпов К.В.\\
				\large {Группа:} P3335\\
				
				\large \textbf{Преподаватель:}\\
				\large Третьяков С.Д.

			\end{flushleft}
		\end{minipage}
	\end{flushright}
	
	\vfill % заполнить всё доступное ниже пространство

	\begin{center}
	\large Санкт-Петербург\\
	\large \the\year % вывести дату
	\end{center} % закончить выравнивание по центру

\thispagestyle{empty} % не нумеровать страницу
\end{titlepage} % конец титульной страницы

\vfill % заполнить всё доступное ниже пространство


% Содержание
\include{ToC}


\section{Цель работы}
Изучение команд для скачивания файлов из Интернета. Изучение команд для работы с архивами. Изучение команд для поиска файлов и слов в файлах.

\section{Задание 1 Скачивание файлов из интернета с использованием терминала:}

\subsection{wget - команда позволяет загружать файлы из сети Интернет. Поддерживает протоколы HTTP, FTP и HTTPS, а также поддерживает работу через HTTP прокси-сервер.}

\begin{figure}[H]
	\begin{center}
		\includegraphics[scale=0.5]{1_1}
		\caption{Отображаем справку команды wget} 
		\label{pic:pic_1} % название для ссылок внутри кода
	\end{center}
\end{figure}

\subsubsection{Параметр --spider проверяет ссылки на доступность}
\subsubsection{Параметр -i даёт возможность работать со ссылками из файлов}

\begin{figure}[H]
	\begin{center}
		\includegraphics[scale=0.5]{1_2}
		\caption{Проверяем доступность ссылок находящихся в файле} 
		\label{pic:pic_1} % название для ссылок внутри кода
	\end{center}
\end{figure}

\subsubsection{Описание - после команды wget следует ссылка на файл.}

\begin{figure}[H]
	\begin{center}
		\includegraphics[scale=0.5]{1_3}
		\caption{Скачиваем первый в списке доступный файл} 
		\label{pic:pic_1} % название для ссылок внутри кода
	\end{center}
\end{figure}

\begin{figure}[H]
	\begin{center}
		\includegraphics[scale=0.5]{1_4}
		\caption{Создаём в домашней папке директорию lab2} 
		\label{pic:pic_1} % название для ссылок внутри кода
	\end{center}
\end{figure}
 
\subsubsection{Параметр -P позволяет сохранять файлы в каталог}
\subsubsection{Параметр -O позволяет менять имя скаченного файла}

\begin{figure}[H]
	\begin{center}
		\includegraphics[scale=0.5]{1_5}
		\caption{Скачиваем второй файл в lab2, изменяем имя на 'var1'} 
		\label{pic:pic_1} % название для ссылок внутри кода
	\end{center}
\end{figure} 
 
\begin{figure}[H]
	\begin{center}
		\includegraphics[scale=0.5]{1_6}
		\caption{Скачиваем остальные файлы в директорию lab2} 
		\label{pic:pic_1} % название для ссылок внутри кода
	\end{center}
\end{figure}

\subsubsection{Параметр -r позволяет рекурсивно загружать файлы с сайта}
\subsubsection{Параметр --level=1 устанавливает глубину рекурсии равную единице. Это значит, будут скачиваться файлы находящиеся на данной странице, без перехода в глубь}
\subsubsection{Параметры -A jpeg jpg позволяют скачать файлы с соответствующими форматами}
\subsubsection{Парметр -nd запрещает создавать директории}

\begin{figure}[H]
	\begin{center}
		\includegraphics[scale=0.5]{1_7_1}
		\caption{Скачиваем все .jpeg и .jpg файлы с сайта. Часть 1} 
		\label{pic:pic_1} % название для ссылок внутри кода
	\end{center}
\end{figure}

\begin{figure}[H]
	\begin{center}
		\includegraphics[scale=0.5]{1_7_2}
		\caption{Скачиваем все .jpeg и .jpg файлы с сайта. Часть 2} 
		\label{pic:pic_1} % название для ссылок внутри кода
	\end{center}
\end{figure}

\newpage

\section{Задание 2. Работа с архивами:}

\subsection{unzip - утилита позволяет извелкать файлы из архивов}

\begin{figure}[H]
	\begin{center}
		\includegraphics[scale=0.5]{2_2}
		\caption{С помощью команды ls проверяем форматы архивов} 
		\label{pic:pic_1} % название для ссылок внутри кода
	\end{center}
\end{figure}

\subsubsection{Параметр -p позволяет извлечь файлы без сообщений}

\begin{figure}[H]
	\begin{center}
		\includegraphics[scale=0.5]{2_3}
		\caption{Извлекаем архив} 
		\label{pic:pic_1} % название для ссылок внутри кода
	\end{center}
\end{figure}

\subsubsection{Параметры -xf позволяют извлечь содержимое архива}
\subsubsection{Параметр -v повзоляет выводить информацию о выполняемых действиях}

\begin{figure}[H]
	\begin{center}
		\includegraphics[scale=0.5]{2_4_1}
		\caption{C помощью утилиты tar извлечем все файлы из архивов} 
		\label{pic:pic_1} % название для ссылок внутри кода
	\end{center}
\end{figure}

\newpage

\section{Задание 3. Поиск файлов, поиск по тексту:}

\subsection{find - утилита поиска файлов по имени и другим свойствам, используемая в Unix-подобных операционных системах}

\subsubsection{Параметр -maxdeth указывает глубину проверки}
\subsubsection{Параметр -mtime ищет файлы, с которыми производились действия меньше дня назад}

\begin{figure}[H]
	\begin{center}
		\includegraphics[scale=0.5]{3_1}
		\caption{Найти все файлы, созданные в домашней папке за последний день} 
		\label{pic:pic_1} % название для ссылок внутри кода
	\end{center}
\end{figure}

\subsubsection{Параметр -name позволяет искать файлы по имени}

\begin{figure}[H]
	\begin{center}
		\includegraphics[scale=0.5]{3_2}
		\caption{Найти все файлы c фамилией автора} 
		\label{pic:pic_1} % название для ссылок внутри кода
	\end{center}
\end{figure}

\subsubsection{Параметр -exec используется для указания действий, которые будут выполнены при нахождении файла с именем, удовлетворяющим выражению поиска}

\begin{figure}[H]
	\begin{center}
		\includegraphics[scale=0.5]{3_3}
		\caption{Находим и перемещаем все *.jpg файлы} 
		\label{pic:pic_1} % название для ссылок внутри кода
	\end{center}
\end{figure}

\begin{figure}[H]
	\begin{center}
		\includegraphics[scale=0.5]{3_4}
		\caption{Переносим .txt файл} 
		\label{pic:pic_1} % название для ссылок внутри кода
	\end{center}
\end{figure}

\begin{figure}[H]
	\begin{center}
		\includegraphics[scale=0.5]{3_5}
		\caption{Создаем директория Картинки с помощью команды mkdir. После производим поиск по нужному формату и переносим их в папку картинки.} 
		\label{pic:pic_1} % название для ссылок внутри кода
	\end{center}
\end{figure} 

\subsection{grep - утилита командной строки, которая находит на вводе строки, отвечающие заданному регулярному выражению, и выводит их, если вывод не отменён специальным ключом.}
\subsubsection{Параметр -с отключает стандартный способ вывода результата и вместо этого отображает только число обозначающее количество найденых строк}
\subsubsection{Параметр -i делает поиск регистронезависимым}

\begin{figure}[H]
	\begin{center}
		\includegraphics[scale=0.5]{3_}
		\caption{Подсчёт количества слов вида "то*"} 
		\label{pic:pic_1} % название для ссылок внутри кода
	\end{center}
\end{figure}

\section{Задание 4. Работа с архивами:}

\subsection{zip - утилита сжатия}

\subsubsection{Параметр -r добавляет все файлы в архив и указываем название архива и нужную папку}

\begin{figure}[H]
	\begin{center}
		\includegraphics[scale=0.5]{4_1}
		\caption{Архивируем, используя zip} 
		\label{pic:pic_1} % название для ссылок внутри кода
	\end{center}
\end{figure}

\subsection{tar - наиболее распространенный архиватор, используемый в Linux-системах}

\subsubsection{Параметр -с служит для создания архива}
\subsubsection{Параметр -v необходим для расширенного вывода информации о выполненных действиях}
\subsubsection{Параметр -f выводит информацию извлекаемую из файла}

\begin{figure}[H]
	\begin{center}
		\includegraphics[scale=0.5]{4_2}
		\caption{Архивируем картинки, используя tar} 
		\label{pic:pic_1} % название для ссылок внутри кода
	\end{center}
\end{figure}

\subsection{tgz - архивирует нужную нам папку через два архиватора tar и gzip}

\begin{figure}[H]
	\begin{center}
		\includegraphics[scale=0.5]{4_4}
		\caption{Архивируем картинки, используя tar} 
		\label{pic:pic_1} % название для ссылок внутри кода
	\end{center}
\end{figure}

\subsection{Просмотр содержания архивов}

\begin{figure}[H]
	\begin{center}
		\includegraphics[scale=0.3]{4_5}
		\caption{Заархивированный .txt файл} 
		\label{pic:pic_1} % название для ссылок внутри кода
	\end{center}
\end{figure}

\begin{figure}[H]
	\begin{center}
		\includegraphics[scale=0.3]{4_6}
		\caption{Заархивированные картинки} 
		\label{pic:pic_1} % название для ссылок внутри кода
	\end{center}
\end{figure}

\section{Задание 5. Просмотр содержимого файлов:}

\begin{figure}[H]
	\begin{center}
		\includegraphics[scale=0.5]{5_1}
		\caption{Расширения файла в папке "Произведения Лермонтова" и файла извлеченного из .zip архива .txt} 
		\label{pic:pic_1} % название для ссылок внутри кода
	\end{center}
\end{figure}

\subsection{cat - утилита последовательно читает файлы и пишет их в стандартный вывод}

\subsubsection{Параметр -v выводит строки которые не удовлетворяют ключу}
\subsubsection{Параметр -n нумерует строки}

\begin{figure}[H]
	\begin{center}
		\includegraphics[scale=0.5]{5_2}
		\caption{С помощью утилиты cat выводим содержимое файла без пустых строк, нумеруем строки} 
		\label{pic:pic_1} % название для ссылок внутри кода
	\end{center}
\end{figure}

\subsection{more - утилита позволяет просмотреть содержимое файла. С помощью параметра -n можно вывести определенное количество строк на экран}

\begin{figure}[H]
	\begin{center}
		\includegraphics[scale=0.5]{5_6}
		\caption{Вывод текста} 
		\label{pic:pic_1} % название для ссылок внутри кода
	\end{center}
\end{figure}

\subsection{Утилита less, как и утилита more, используется для просмотра содержимого файлов}

\begin{figure}[H]
	\begin{center}
		\includegraphics[scale=0.5]{5_4}
		\caption{Команда} 
		\label{pic:pic_1} % название для ссылок внутри кода
	\end{center}
\end{figure}
\begin{figure}[H]
	\begin{center}
		\includegraphics[scale=0.5]{5_5}
		\caption{Вывод текста} 
		\label{pic:pic_1} % название для ссылок внутри кода
	\end{center}
\end{figure}

\section{Задание 6. Управление правами доступа к файлам и каталогам:}

Размер файла в папке "Произведения Лермонтова" больше по сравнению с файлом в папке "lab2". Оба файла имеют одинаковые права. Владелец имеет права на чтение и запись, группа и все остальные имеют права только на чтение

\begin{figure}[H]
	\begin{center}
		\includegraphics[scale=0.5]{6_1}
		\caption{} 
		\label{pic:pic_1} % название для ссылок внутри кода
	\end{center}
\end{figure}

\subsection{Удаляем права на чтение с помощью утилиты chmod, которая позволяет менять права доступа к директориям и файлам}

\subsubsection{Параметры go-r означают, что для группы и остальных, удаляются права на чтение. С помощью утилиты less имеется возможность просмотреть содержимое файла, так как эти действия производятся владельцем файла}

\begin{figure}[H]
	\begin{center}
		\includegraphics[scale=0.5]{6_2}
		\caption{} 
		\label{pic:pic_1} % название для ссылок внутри кода
	\end{center}
\end{figure}

\subsubsection{Параметры u-w означают, что для владельца файла удаляется право на запись}

\begin{figure}[H]
	\begin{center}
		\includegraphics[scale=0.5]{6_3_1}
		\caption{} 
		\label{pic:pic_1} % название для ссылок внутри кода
	\end{center}
\end{figure}

\subsection{Добаляем права на запуск для группы и остальных пользователей. Запуск файла невозможен, так как у владельца файла отсутствуют права на запуск}

\begin{figure}[H]
	\begin{center}
		\includegraphics[scale=0.5]{6_4}
		\caption{} 
		\label{pic:pic_1} % название для ссылок внутри кода
	\end{center}
\end{figure}

\subsection{С помощью утилиты echo добавляем строчку в файл, проверяем права доступа. Владелец и группа пользователей имеют права на чтение и запись, остальные имеют права только на чтение. Добавляем права на запуск для владельца и группы в восьмиричной системе. Запуск удачен}

\begin{figure}[H]
	\begin{center}
		\includegraphics[scale=0.5]{6_5}
		\caption{} 
		\label{pic:pic_1} % название для ссылок внутри кода
	\end{center}
\end{figure}

\section{Задание 6. Команды для ввода/вывода данных. Перенаправление ввода/вывода:}

\subsection{С помощью команды chmod добавляем права на выполнение}

\begin{figure}[H]
	\begin{center}
		\includegraphics[scale=0.5]{7_1}
		\caption{} 
		\label{pic:pic_1} % название для ссылок внутри кода
	\end{center}
\end{figure}

\subsection{Перенаправляем вывод в файл message.txt и ошибки в файл error.txt. Файл с сообщениями содрежит строчку 'Hello World!', а файл с ошибками пуст}

\begin{figure}[H]
	\begin{center}
		\includegraphics[scale=0.5]{7_2}
		\caption{} 
		\label{pic:pic_1} % название для ссылок внутри кода
	\end{center}
\end{figure}

\subsection{Используем входной поток с командой cat. Получаем строчку 'Hello World!' в терминале}

\begin{figure}[H]
	\begin{center}
		\includegraphics[scale=0.5]{7_3}
		\caption{} 
		\label{pic:pic_1} % название для ссылок внутри кода
	\end{center}
\end{figure}

\section{Вывод}
В данной лабораторной работе были изучены команды для скачивания файлов из Интернета, для работы с архивами, а также для поиска файлов и слов в файлах. 
\end{document}
