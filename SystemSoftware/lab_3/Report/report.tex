\include{settings}

\begin{document}	% начало документа

% Титульная страница
\begin{titlepage}	% начало титульной страницы

	\begin{center}		% выравнивание по центру

		\large Санкт-Петербургский Национально Исследовательский Университет\\
		\large информационных технологий, механики и оптики \\
		\large Кафедра систем управления и информатики\\[6cm]
		% название института, затем отступ 6см
		
		\huge Технология изготовления элементов приборов и систем\\[0.5cm] % название работы, затем отступ 0,5см
		\large Отчет по лабораторной работе №5\\[0.1cm]
		\large Расчет режима резания при фрезеровании\\[1cm]
		\large Вариант №2\\[5cm]

	\end{center}


	\begin{flushright} % выравнивание по правому краю
		\begin{minipage}{0.25\textwidth} % врезка в половину ширины текста
			\begin{flushleft} % выровнять её содержимое по левому краю

				\large\textbf{Работу выполнили:}\\
				\large Зенкин А.М.\\
				\large Карпов К.В.\\
				\large {Группа:} P3335\\
				
				\large \textbf{Преподаватель:}\\
				\large Третьяков С.Д.

			\end{flushleft}
		\end{minipage}
	\end{flushright}
	
	\vfill % заполнить всё доступное ниже пространство

	\begin{center}
	\large Санкт-Петербург\\
	\large \the\year % вывести дату
	\end{center} % закончить выравнивание по центру

\thispagestyle{empty} % не нумеровать страницу
\end{titlepage} % конец титульной страницы

\vfill % заполнить всё доступное ниже пространство


% Содержание
\include{ToC}


\section{Цель работы}
Написание скриптов на bash.

\section{Ход работы}

\subsection{Задание 1 «Написание скриптов. Операторы ветвления»}

\subsubsection{Знакомство с командной оболочкой bash}
	\begin{figure}[H]
		\begin{center}
			\includegraphics[scale=0.5]{1}
			\caption{- Отображаем справку по bash} 
			\label{pic:pic_1} % название для ссылок внутри кода
		\end{center}
	\end{figure}

\subsubsection{Bash-скрипт, который выводит Ф.И.О. и вариант группы тремя способами:}
	\lstinputlisting[
		label=code:plot,
		caption={Ф.И.О},
	]{firstScript.sh}
	\parindent=1cm

\subsubsection{Анализ каждого метода из пункта 2.1.2:}
1. Преимущество: код меньший по объёму. Недостатки: выводится текст с которым никак нельзя взаимодействовать.\\

2. Достоинства в том, что можно производить операции с этой строковой переменной. Недостаток, что занимает память.\\

3. Достоинство заключается в том, что можно использовать любое название переменной, т.к. она существует только внутри функции. Недостаток в том, что код может стать непонятным.

\subsubsection{Использование операторов ветвления:}
	\lstinputlisting[
		label=code:plot,
		caption={проверка на число},
	]{secondScript.sh}
	\parindent=1cm

\newpage

\subsection{Задание 2 «Написание скриптов на bash. Циклы»}
\subsubsection{Использование циклов:}
	\lstinputlisting[
		label=code:plot,
		caption={группы},
	]{firdScript.sh}
	\parindent=1cm

\subsubsection{Факториал n для заданного n, принадлежащего R:}
	\lstinputlisting[
		label=code:plot,
		caption={факториал},
	]{fourthScript.sh}
	\parindent=1cm

\section{Вывод}
В данной лабораторной работе были получены навыки в написании скриптов на bash - освнов программирования, циклов и операторов ветвления. 
\end{document}
